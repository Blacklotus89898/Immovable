\documentclass[a4paper,12pt]{article}

% Packages
\usepackage[utf8]{inputenc}
\usepackage[T1]{fontenc}
\usepackage{lmodern}
\usepackage{geometry}
\usepackage{hyperref}
\hypersetup{
    colorlinks=true,
    linkcolor=blue,
    urlcolor=cyan,
    pdftitle={Programming Language Analysis Report},
    pdfauthor={Stev Chen}
}
\geometry{margin=1in}
\setlength{\parskip}{1em}
\setlength{\parindent}{0em}

% Title & Metadata
\title{Programming Language Analysis Report: Rust}
\author{Steve Chen and GPT}
\date{\today}

\begin{document}

\maketitle
\tableofcontents
\newpage

% -------------------- SECTION 1 --------------------
\section*{Document Metadata}
\addcontentsline{toc}{section}{Document Metadata}

\begin{tabular}{|l|p{10cm}|}
\hline
\textbf{Language Name} & Rust \\
\hline
\textbf{Author(s)} & Steve Chen \\
\hline
\textbf{Date} & \today \\
\hline
\textbf{Version} & 1.0 \\
\hline
\end{tabular}

% -------------------- SECTION 2 --------------------
\section{Executive Summary}

\subsection{Overview}
A high-level introduction to the Rust programming language.

\subsection{Purpose of This Report}
Describe the aim of the analysis — educational, evaluative, etc.

\subsection{Key Findings}
Summarize Rust’s core strengths, market position, and future prospects.

% -------------------- SECTION 3 --------------------
\section{Background and Context}

\subsection{History}
Origin, first release, and historical motivation behind Rust.

\subsection{Creators and Organization}
The Mozilla Foundation and the Rust community’s involvement.

\subsection{Evolution and Milestones}
Versioning, tooling improvements, and major releases.

\subsection{Typical Use Cases}
Systems programming, embedded, web (via WASM), CLI tools, etc.

% -------------------- SECTION 4 --------------------
\section{Language Philosophy}

\subsection{Core Design Principles}
Memory safety without garbage collection, zero-cost abstractions, fearless concurrency.

\subsection{Programming Paradigms}
Supports functional, imperative, and concurrent programming styles.

\subsection{Typing System}
Static, strong, and type-inferred.

% -------------------- SECTION 5 --------------------
\section{Syntax and Semantics}

\subsection{Hello World Example}
\begin{verbatim}
fn main() {
    println!("Hello, world!");
}
\end{verbatim}

\subsection{Variables}
Immutable by default with `let`; mutable via `let mut`.

\subsection{Functions}
Function declaration syntax and return types.

\subsection{Control Structures}
if/else, match, loop, while, for.

\subsection{Modules and Namespaces}
`mod`, `use`, and `crate` system for package organization.

\subsection{Standard Library Overview}
Brief look at `std::collections`, `std::io`, and `std::thread`.

% -------------------- SECTION 6 --------------------
\section{History and Evolution}
\subsection{Key Milestones}
\subsection{Major Contributors}

% -------------------- SECTION 7 --------------------
\section{Advantages and Limitations}
\subsection{Strengths}
\subsection{Challenges and Drawbacks}

% -------------------- SECTION 8 --------------------
\section{Conclusion}
\subsection{Summary of Findings}
\subsection{Future Potential and Trends}

\end{document}
